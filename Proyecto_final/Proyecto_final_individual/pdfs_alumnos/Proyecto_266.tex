\documentclass{article}%
\usepackage[T1]{fontenc}%
\usepackage[utf8]{inputenc}%
\usepackage{lmodern}%
\usepackage{textcomp}%
\usepackage{lastpage}%
\usepackage[utf8]{inputenc}%
\usepackage[T1]{fontenc}%
\usepackage{amsmath, amssymb, setspace, fancyhdr}%
%
\pagestyle{fancy}%
\fancyhf{}%
\lhead{Facultad de Ingeniería Tampico – UAT}%
\rhead{Proyecto Final de Álgebra Lineal}%
\setlength{\headheight}{16pt}%
%
\begin{document}%
\12pt%
% ======================================================
% Archivo: instrucciones_y_ejemplo.tex
% ======================================================
\section*{Instrucciones del proyecto}

Este proyecto tiene como objetivo aplicar los conceptos de álgebra lineal al proceso de
\textbf{cifrado y descifrado de mensajes mediante matrices}.
Cada alumno recibe una \textbf{matriz llave $K$} y una \textbf{cadena de números cifrados}.
Su tarea consiste en:

\begin{enumerate}
    \item Calcular la matriz inversa $K^{-1}$ en el cuerpo $\mathbb{Z}_{29}$.
    \item Multiplicar $K^{-1}$ por los vectores de la cadena cifrada (en bloques de 3 en 3 números).
    \item Obtener la secuencia numérica original y convertirla a texto según la tabla de equivalencias.
\end{enumerate}

El mensaje resultante será una frase de aproximadamente 100 caracteres.
Presente todos los cálculos y procedimientos en el espacio indicado.

\bigskip
\section*{Ejemplo de descifrado}

Suponga que se le da la siguiente matriz y cadena cifrada:

\[
K =
\begin{pmatrix}
2 & 5 & 7 \\
1 & 6 & 3 \\
4 & 0 & 8
\end{pmatrix},
\quad
\text{Cadena cifrada: } [7,\, 18,\, 3,\, 4,\, 9,\, 2,\, 15,\, 21,\, 5]
\]

\begin{enumerate}
    \item Calcule $K^{-1} \pmod{29}$.
    \item Agrupe los números en vectores columna de tamaño 3: $(7,18,3)$, $(4,9,2)$, $(15,21,5)$.
    \item Multiplique $K^{-1}$ por cada vector y obtenga el mensaje original.
    \item Convierta los números resultantes a letras con la tabla del curso.
\end{enumerate}

\bigskip
\hrule
\bigskip%

\textbf{Proyecto 266}\\[2pt]
\hrule
\bigskip

\textbf{Nombre del alumno:} \underline{\hspace{8cm}}\\[8pt]
\textbf{Matrícula:} \underline{\hspace{4cm}} \hspace{1cm}
\textbf{Grupo:} \underline{\hspace{2cm}} \hspace{1cm}
\textbf{Fecha de entrega:} \underline{\hspace{2.5cm}}\\[12pt]

\textbf{Matriz llave:}
\[
K = \begin{pmatrix}
15 & 1 & 5 \\
25 & 4 & 5 \\
8 & 7 & 1
\end{pmatrix} \pmod{29}
\]

\textbf{Cadena cifrada:}
\begin{center}
$\begin{array}{lllllllllllllll}
16 & 7 & 19 & 16 & 10 & 7 & 12 & 0 & 7 & 14 & 2 & 26 & 23 & 5 & 9\\
15 & 12 & 14 & 23 & 1 & 19 & 13 & 2 & 13 & 8 & 16 & 21 & 21 & 7 & 28\\
23 & 2 & 27 & 2 & 10 & 21 & 18 & 26 & 1 & 18 & 25 & 24 & 5 & 26 & 5\\
23 & 5 & 9 & 28 & 25 & 5 & 10 & 23 & 3 & 2 & 10 & 21 & 2 & 10 & 16\\
18 & 19 & 3 & 4 & 13 & 12 & 23 & 3 & 25 & 27 & 21 & 14 & 18 & 17 & 4\\
12 & 16 & 25 & 6 & 23 & 7 & 19 & 13 & 18 & 25 & 5 & 24 & 15 & 1 & 18\\
25 & 17 & 6 & 7 & 25 & 23 & 1 & 5 & 21 & 0 & 7 & 11\\
\end{array}$
\end{center}

\bigskip
\textbf{Espacio para cálculos y observaciones:}\\[6pt]
\rule{\linewidth}{0.4pt}\\[10pt]
\rule{\linewidth}{0.4pt}\\[10pt]
\rule{\linewidth}{0.4pt}\\[10pt]
\rule{\linewidth}{0.4pt}\\[10pt]
%
\end{document}