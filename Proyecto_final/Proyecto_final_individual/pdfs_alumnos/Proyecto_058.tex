\documentclass{article}%
\usepackage[T1]{fontenc}%
\usepackage[utf8]{inputenc}%
\usepackage{lmodern}%
\usepackage{textcomp}%
\usepackage{lastpage}%
\usepackage[a4paper, margin=2.5cm, headheight=16pt]{geometry}%
\usepackage[spanish,es-noshorthands]{babel}%
\usepackage[T1]{fontenc}%
\usepackage{amsmath, amssymb, setspace, fancyhdr}%
%
\renewcommand\normalsize{\fontsize{12}{14}\selectfont}%
\normalsize%
\pagestyle{fancy}%
\fancyhf{}%
\lhead{Facultad de Ingeniería Tampico – UAT}%
\rhead{Proyecto Final de Álgebra Lineal}%
\setlength{\headheight}{16pt}%
%
\begin{document}%
\normalsize%
% ======================================================
% Archivo: instrucciones_y_ejemplo.tex
% ======================================================
\section*{Instrucciones del proyecto}

Este proyecto tiene como objetivo aplicar los conceptos de álgebra lineal al proceso de
\textbf{cifrado y descifrado de mensajes mediante matrices}.
Cada alumno recibe una \textbf{matriz llave $K$} y una \textbf{cadena de números cifrados}.
Su tarea consiste en:

\begin{enumerate}
    \item Calcular la matriz inversa $K^{-1}$ utilizando el \textbf{método de Gauss–Jordan}.
    \item Multiplicar la matriz inversa $K^{-1}$ por los vectores de la cadena cifrada (en bloques de 3 en 3 números).
    \item Obtener la secuencia numérica original y convertirla a texto según la tabla de equivalencias proporcionada.
\end{enumerate}

El mensaje resultante corresponderá a una frase corta que deberá descifrarse correctamente.
Presente todos los cálculos y procedimientos paso a paso en el espacio indicado.

\bigskip

\section*{Ejemplo de descifrado}

Suponga que se le da la siguiente matriz y cadena cifrada:

\[
K =
\begin{pmatrix}
2 & 5 & 7 \\
1 & 6 & 3 \\
4 & 0 & 8
\end{pmatrix},
\quad
\text{Cadena cifrada: } [7,\, 18,\, 3,\, 4,\, 9,\, 2,\, 15,\, 21,\, 5]
\]

\begin{enumerate}
    \item \textbf{Calcular la matriz inversa $K^{-1}$ utilizando el método de Gauss–Jordan.}

    Para ello, se forma la matriz aumentada:

    \[
    [K \,|\, I] =
    \begin{pmatrix}
    2 & 5 & 7 & 1 & 0 & 0 \\
    1 & 6 & 3 & 0 & 1 & 0 \\
    4 & 0 & 8 & 0 & 0 & 1
    \end{pmatrix}
    \]

    Luego, aplicando operaciones elementales de fila (intercambio, multiplicación y suma), se transforma la parte izquierda en la identidad.
    El resultado final es:

    \[
    [I \,|\, K^{-1}] =
    \begin{pmatrix}
    1 & 0 & 0 & 0.50 & -0.39 & -0.22 \\
    0 & 1 & 0 & -0.10 & 0.26 & -0.09 \\
    0 & 0 & 1 & -0.25 & 0.24 & 0.18
    \end{pmatrix}
    \quad \Rightarrow \quad
    K^{-1} =
    \begin{pmatrix}
    0.50 & -0.39 & -0.22 \\
    -0.10 & 0.26 & -0.09 \\
    -0.25 & 0.24 & 0.18
    \end{pmatrix}
    \]

    \item \textbf{Agrupar la cadena cifrada en vectores de tamaño 3:}
    \[
    (7,18,3), \quad (4,9,2), \quad (15,21,5)
    \]

    \item \textbf{Multiplicar $K^{-1}$ por cada vector} para recuperar los números originales del mensaje.
    Por ejemplo, para el primer bloque:

    \[
    \begin{pmatrix}
    0.50 & -0.39 & -0.22 \\
    -0.10 & 0.26 & -0.09 \\
    -0.25 & 0.24 & 0.18
    \end{pmatrix}
    \begin{pmatrix}
    7 \\ 18 \\ 3
    \end{pmatrix}
    =
    \begin{pmatrix}
    0.50(7) - 0.39(18) - 0.22(3) \\
    -0.10(7) + 0.26(18) - 0.09(3) \\
    -0.25(7) + 0.24(18) + 0.18(3)
    \end{pmatrix}
    =
    \begin{pmatrix}
    0.83 \\ 3.03 \\ 1.45
    \end{pmatrix}
    \approx
    \begin{pmatrix}
    1 \\ 3 \\ 1
    \end{pmatrix}
    \]

    Repitiendo este proceso para los demás bloques, se obtienen los números descifrados.

    \item \textbf{Convertir los números a letras} utilizando la siguiente tabla de equivalencias:

    \begin{center}
    \renewcommand{\arraystretch}{1.2}
    \small
    \begin{tabular}{cccccc}
    A=1 & B=2 & C=3 & D=4 & E=5 & F=6 \\
    G=7 & H=8 & I=9 & J=10 & K=11 & L=12 \\
    M=13 & N=14 & O=15 & P=16 & Q=17 & R=18 \\
    S=19 & T=20 & U=21 & V=22 & W=23 & X=24 \\
    Y=25 & Z=26 & Espacio=27 & ,=28 & .=29 &
    \end{tabular}
    \end{center}

    \item \textbf{Interpretar el mensaje obtenido.}

    Supongamos que el resultado final es:

    \[
    [3,\, 15,\, 4,\, 9,\, 7,\, 15,\, 27,\, 19,\, 5,\, 3,\, 18,\, 5,\, 20,\, 15]
    \]

    Usando la tabla anterior:

    \[
    \text{C O D I G O (espacio) S E C R E T O}
    \]

    Por lo tanto, el mensaje descifrado es:

    \[
    \boxed{\text{CODIGO SECRETO}}
    \]
\end{enumerate}

\bigskip
\textit{Nota: el propósito de este ejemplo es ilustrar el procedimiento paso a paso del método de Gauss–Jordan.
Cada alumno deberá aplicar el mismo proceso con su propia matriz y cadena cifrada.}

\bigskip
\hrule
\bigskip%

\newpage
\vspace{0.5em}
\noindent\textbf{Proyecto 058}\\%\\[0.5cm]
\noindent\hrule
\vspace{1em}

\noindent\textbf{Nombre del alumno:} \underline{\hspace{11.8cm}}\\[8pt]
\noindent\textbf{Matrícula:} \underline{\hspace{4cm}} 
\textbf{Grupo:} \underline{\hspace{1.9cm}}
\textbf{Fecha de entrega:} \underline{\hspace{2.5cm}}\\[12pt]

\textbf{Matriz llave:}
\[
K = \begin{pmatrix}
6.0 & 1.0 & 1.0 \\
5.0 & 1.0 & 3.0 \\
6.0 & 5.0 & 3.0
\end{pmatrix} \pmod{29}
\]

\textbf{Cadena cifrada:}
\begin{center}
$\begin{array}{lllllllllllllll}
100.0 & 142.0 & 158.0 & 147.0 & 166.0 & 191.0 & 180.0 & 183.0 & 222.0 & 99.0 & 103.0 & 141.0 \\
38.0 & 47.0 & 156.0 & 131.0 & 157.0 & 279.0 & 173.0 & 192.0 & 321.0 & 23.0 & 32.0 & 81.0 \\
119.0 & 135.0 & 229.0 & 179.0 & 154.0 & 245.0 & 136.0 & 172.0 & 194.0 & 132.0 & 152.0 & 190.0 \\
124.0 & 153.0 & 216.0 & 75.0 & 124.0 & 201.0 & 133.0 & 155.0 & 193.0 & 136.0 & 175.0 & 266.0 \\
47.0 & 72.0 & 135.0 & 181.0 & 156.0 & 255.0 & 185.0 & 187.0 & 273.0 & 28.0 & 45.0 & 98.0 \\
64.0 & 99.0 & 160.0 & 142.0 & 177.0 & 300.0 & 175.0 & 166.0 & 209.0 & 126.0 & 125.0 & 156.0 \\
117.0 & 119.0 & 183.0 & 112.0 & 152.0 & 170.0 & 25.0 & 38.0 & 87.0 & 117.0 & 120.0 & 207.0 \\
160.0 & 194.0 & 266.0 & 47.0 & 71.0 & 111.0 & 28.0 & 45.0 & 98.0 & 46.0 & 97.0 & 104.0 \\
117.0 & 107.0 & 145.0 & 192.0 & 223.0 & 254.0
\end{array}$
\end{center}

\bigskip
\textbf{Espacio para cálculos y observaciones:}\\[6pt]
\rule{\linewidth}{0.4pt}\\[10pt]
\rule{\linewidth}{0.4pt}\\[10pt]
\rule{\linewidth}{0.4pt}\\[10pt]
\rule{\linewidth}{0.4pt}\\[10pt]
%
\end{document}