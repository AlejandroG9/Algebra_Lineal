\documentclass[12pt,a4paper]{article}

% Idioma y codificación
\usepackage[spanish,es-nodecimaldot]{babel}
\usepackage[utf8]{inputenc}
\usepackage[T1]{fontenc}

% Matemáticas y formato
\usepackage{amsmath, amssymb, amsfonts}
\usepackage{geometry}
\usepackage{fancyhdr}
\usepackage{enumitem}
\usepackage{hyperref}
\usepackage{xcolor}
\usepackage{booktabs}
\usepackage{multicol}

% Macros del proyecto
\usepackage{macros}

\geometry{margin=2.5cm}
\pagestyle{fancy}
\fancyhf{}
\rhead{Álgebra Lineal}
\lhead{Problemarios}
\rfoot{\thepage}

\title{\Huge Problemarios de Álgebra Lineal\\[4pt]
\Large Módulo 1: Números Imaginarios Puros}
\vspace{0.5cm}
\author{%
Facultad de Ingeniería Tampico -- UAT \\
\vspace{0.5cm}
\textbf{Autores:} \\
Dr. Alejandro González Turrubiates \\
Dr. Marcos Alfredo Azuara Hernandez\\
Dr. Juan Enrique Bermea Barrios
}
\date{\today}

\begin{document}
\maketitle
\thispagestyle{empty}
\newpage

% ======= SECCIONES =======
%! Author = alejandrogonzalezturrubiates
%! Date = 18/08/25

\section*{Módulo 1: Números Imaginarios Puros}

\begin{objetivos}
  \item Comprender la unidad imaginaria $\ii$ con $\ii^2=-1$.
  \item Convertir raíces negativas a expresiones con $\ii$.
  \item Resolver suma, resta, multiplicación y división con imaginarios puros.
  \item Usar potencias de $\ii$ mediante periodicidad mod 4.
\end{objetivos}

\begin{marcoteorico}
  \item Definición: $\ii=\sqrt{-1}$.
  \item Conversión: $\sqrt{-a}=\sqrt{a}\,\ii$ para $a>0$.
  \item Potencias: $\ii^1=\ii,\ \ii^2=-1,\ \ii^3=-\ii,\ \ii^4=1,\ \ii^{n}=\ii^{\,n\bmod 4}$.
\end{marcoteorico}
%! Author = alejandrogonzalezturrubiates
%! Date = 18/08/25

\begin{ejemplos}
  \item \(\sqrt{-16}=4\ii\).
  \item \(\sqrt{-a^2}=a\ii\) (con \(a\ge 0\)).
  \item \(\sqrt{-4}+\sqrt{-9}=2\ii+3\ii=5\ii\).
\end{ejemplos}

\begin{actividad}{1: Conversión a imaginarios}
  \item \(\sqrt{-2}\)
  \item \(\sqrt{-9}\)
  \item \(\sqrt{-81}\)
  \item \(\sqrt{-b^2}\)
  \item \(\sqrt{-4m^4}\)
  \item \(\sqrt{-a^2-b^2}\)
\end{actividad}
%! Author = alejandrogonzalezturrubiates
%! Date = 18/08/25

% Preamble
\begin{ejemplos}
  \item \(\sqrt{-36}=6\ii\), \(\sqrt{-25}=5\ii\) \(\Rightarrow\ 6\ii-5\ii=\ii\).
  \item \(12\ii+(-5\ii)+\sqrt{-12}=12\ii-5\ii+2\sqrt{3}\,\ii=(7+2\sqrt{3})\ii\).
\end{ejemplos}

\begin{actividad}{2: Suma y resta}
  \item \(\sqrt{-4}+\sqrt{-16}\)
  \item \(3\sqrt{-64}-5\sqrt{-49}+3\sqrt{-121}\)
  \item \(3\sqrt{-a^2}+4\sqrt{-9a^2}-3\sqrt{-4a^2}\)
\end{actividad}
%! Author = alejandrogonzalezturrubiates
%! Date = 18/08/25

\begin{ejemplos}
  \item \(\sqrt{-4}\cdot\sqrt{-9}=(2\ii)(3\ii)=6\ii^2=-6\).
  \item \((3\ii+5\sqrt{2}\,\ii)(2\ii-2\sqrt{2}\,\ii)=
        \ii^2\,(3+5\sqrt{2})(2-2\sqrt{2})= -8.25\) y su valor absoluto \(8.25\).
\end{ejemplos}

\begin{actividad}{3: Multiplicación}
  % Ejercicios previos
  \item \(\sqrt{-16}\cdot\sqrt{-25}\)
  \item \((2\sqrt{-2}+5\sqrt{-3})(\sqrt{-2}-4\sqrt{-3})\)
  \item \(\ii\cdot \ii \cdot \ii,\quad \ii^8,\quad \ii^{15}\)

  % ---- Nuevos ejercicios: Monomios ----
  \item \((7\ii)(-3\ii)\)
  \item \((12\ii)(5\ii)(-2\ii)\)
  \item \((-4\ii)(-6\ii)\)
  \item \((9\ii)(-9\ii)(\ii)\)
  \item \((2\ii)(3\ii)(4\ii)(5\ii)\)

  % ---- Nuevos ejercicios: Polinomios ----
  \item \((2\ii+3\ii)(4\ii+5\ii)\)
  \item \((\ii+2)(\ii-2)\)
  \item \((3\ii-1)(3\ii+1)\)
  \item \((\ii+1)(\ii+2)(\ii+3)\)
  \item \((2\ii-5)(2\ii+5)(\ii)\)
\end{actividad}

\begin{retos}
  \item Simplifica: \((3\ii)(4\ii)\).
  \item Evalúa: \((2\ii)(-7\ii)(5\ii)\).
  \item Determina: \(\ii^{2025}\).
\end{retos}
%! Author = alejandrogonzalezturrubiates
%! Date = 18/08/25

\begin{ejemplos}
  \item \(\dfrac{\sqrt{-84}}{\sqrt{-7}} = \dfrac{84\ii}{7\ii} = 12\).
  \item \(\dfrac{10\sqrt{-36}}{5\sqrt{-4}} = \dfrac{10\cdot 6\ii}{5\cdot 2\ii} = \dfrac{60\ii}{10\ii} = 6\).
\end{ejemplos}

\begin{actividad}{4: División}
  \item \(\dfrac{\sqrt{-27}}{\sqrt{-3}}\)
  \item \(\dfrac{\sqrt{-150}}{\sqrt{-3}}\)
  \item \(\dfrac{10\sqrt{-36}}{5\sqrt{-4}}\)
\end{actividad}
%! Author = alejandrogonzalezturrubiates
%! Date = 18/08/25

\section*{Tareas de consolidación}
\begin{enumerate}[label=\textbf{Tarea \arabic*.}, leftmargin=1.25cm]
  \item Convierte y simplifica:
    \begin{enumerate}[label=\alph*)]
      \item \(\sqrt{-a^2-b^2}\)
      \item \(\sqrt{-12}+\sqrt{-27}-\sqrt{-75}\)
      \item \((4\ii-3\ii)(-2\ii+5\ii)\)
    \end{enumerate}
  \item Potencias de \(\ii\): calcula \(\ii^{35}\), \(\ii^{146}\), \(\ii^{1001}\).
  \item Expresión mixta: \(\dfrac{(3\ii)(2\ii)(-5\ii)}{\sqrt{-25}}\).
\end{enumerate}

\bigskip
\noindent\textbf{Indicaciones:} muestra procedimiento y justifica el uso de \(\ii^{n\bmod 4}\).

% Placeholders de módulos futuros (dejados comentados hasta que los llenemos)
% %! Author = alejandrogonzalezturrubiates
%! Date = 18/08/25

% Preamble
\section*{Módulo 2: Números Complejos (placeholder)}
% Objetivos, teoría de forma a+bi, operaciones, forma polar, etc.
% %! Author = alejandrogonzalezturrubiates
%! Date = 18/08/25

\section*{Módulo 3: División de Polinomios (placeholder)}
% Objetivos, método de Ruffini, división larga, aplicaciones.
% %! Author = alejandrogonzalezturrubiates
%! Date = 18/08/25

\section*{Módulo 4: Matrices (placeholder)}
% Objetivos, operaciones básicas, propiedades.
% %! Author = alejandrogonzalezturrubiates
%! Date = 18/08/25

\section*{Módulo 5: Determinantes (placeholder)}
% Objetivos, cofactores, propiedades, Sarrus/Cofactores.

%! Author = alejandrogonzalezturrubiates
%! Date = 18/08/25

\section*{Rúbrica general de evaluación}
\begin{tabular}{p{0.35\linewidth} p{0.6\linewidth}}
\toprule
\textbf{Criterio} & \textbf{Descripción} \\
\midrule
Procedimiento & Presenta pasos claros, justifica conversiones a \(i\) y uso de potencias (mod 4). \\
Exactitud & Resultados correctos y simplificados. \\
Claridad & Notación adecuada, orden y limpieza. \\
Reflexión & Explica errores comunes y cómo evitarlos. \\
\bottomrule
\end{tabular}

\end{document}