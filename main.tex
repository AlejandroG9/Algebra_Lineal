\documentclass[12pt,a4paper]{article}

% ====== Alinear ecuaciones a la izquierda (opcional) ======
\PassOptionsToPackage{fleqn}{amsmath}
\usepackage{mathtools}
\AtBeginDocument{\setlength{\mathindent}{0pt}}

% ====== Idioma y codificación ======
\usepackage[spanish,es-nodecimaldot]{babel}
\usepackage[utf8]{inputenc}
\usepackage[T1]{fontenc}
\usepackage{lmodern}

% ====== Matemáticas y formato ======
\usepackage[margin=2.5cm]{geometry}
\usepackage{amsmath,amssymb,amsfonts}
\usepackage{siunitx}
\usepackage{physics}
\usepackage{xcolor}
\usepackage{graphicx}
\usepackage{booktabs}
\usepackage{multicol}
\usepackage{enumitem}
\usepackage{hyperref}
\hypersetup{
  colorlinks=true,
  linkcolor=blue!60!black,
  urlcolor=blue!60!black,
  citecolor=blue!60!black
}

% (Opcional) compatibilidad \qty entre physics y siunitx
\AtBeginDocument{\RenewCommandCopy\qty\SI}

% ====== Macros del proyecto (controla soluciones y encabezado) ======
\usepackage[solucion]{macros}  % cambia a [nosolucion] si quieres ocultarlas

% Encabezado superior (izquierda, centro, derecha)
\setupEncabezado{Problemarios}{Módulo 1: Números Imaginarios Puros}{Álgebra Lineal}

% ====== Metadatos ======
\title{\Huge Problemarios de Álgebra Lineal\\[4pt]
\Large Módulo 1: Números Imaginarios Puros}
\author{%
Facultad de Ingeniería Tampico -- UAT\\[4pt]
\textbf{Autores:}\\
Dr.\ Alejandro González Turrubiates\\
Dr.\ Marcos Alfredo Azuara Hernandez\\
Dr.\ Juan Enrique Bermea Barrios
}
\date{\today}

\begin{document}
\maketitle
\thispagestyle{empty}
\newpage

% ======= SECCIONES =======
%! Author = alejandrogonzalezturrubiates
%! Date = 18/08/25

\section*{Módulo 1: Números Imaginarios Puros}

\begin{objetivos}
  \item Comprender la unidad imaginaria $\ii$ con $\ii^2=-1$.
  \item Convertir raíces negativas a expresiones con $\ii$.
  \item Resolver operaciones básicas con imaginarios puros: suma, resta, multiplicación y división.
  \item Usar potencias de $\ii$ mediante su periodicidad módulo $4$.
\end{objetivos}

\begin{marcoteorico}
  \item \textbf{Definición:} $\ii=\sqrt{-1}$.
  \item \textbf{Conversión de raíces negativas:} $\sqrt{-a}=\sqrt{a}\,\ii$ para $a>0$.
  \item \textbf{Potencias de $\ii$:}
  \[
    \ii^1=\ii, \quad \ii^2=-1, \quad \ii^3=-\ii, \quad \ii^4=1,
    \quad \ii^{n}=\ii^{\,n \bmod 4}.
  \]
\end{marcoteorico}
%! Author = alejandrogonzalezturrubiates
%! Date = 18/08/25

\begin{ejemplos}
  \item \(\sqrt{-16}=4\ii\).
  \item \(\sqrt{-a^2}=a\ii\) (con \(a\ge 0\)).
  \item \(\sqrt{-4}+\sqrt{-9}=2\ii+3\ii=5\ii\).
\end{ejemplos}

\begin{actividad}{1: Conversión a imaginarios}
  \item \(\sqrt{-2}\)
  \item \(\sqrt{-9}\)
  \item \(\sqrt{-81}\)
  \item \(\sqrt{-b^2}\)
  \item \(\sqrt{-4m^4}\)
  \item \(\sqrt{-a^2-b^2}\)
\end{actividad}
%! Author = alejandrogonzalezturrubiates
%! Date = 18/08/25

% Preamble
\begin{ejemplos}
  \item \(\sqrt{-36}=6\ii\), \(\sqrt{-25}=5\ii\) \(\Rightarrow\ 6\ii-5\ii=\ii\).
  \item \(12\ii+(-5\ii)+\sqrt{-12}=12\ii-5\ii+2\sqrt{3}\,\ii=(7+2\sqrt{3})\ii\).
\end{ejemplos}

\begin{actividad}{2: Suma y resta}
  \item \(\sqrt{-4}+\sqrt{-16}\)
  \item \(3\sqrt{-64}-5\sqrt{-49}+3\sqrt{-121}\)
  \item \(3\sqrt{-a^2}+4\sqrt{-9a^2}-3\sqrt{-4a^2}\)
\end{actividad}
%! Author = alejandrogonzalezturrubiates
%! Date = 18/08/25

\begin{ejemplos}
  \item \(\sqrt{-4}\cdot\sqrt{-9}=(2\ii)(3\ii)=6\ii^2=-6\).
  \item \((3\ii+5\sqrt{2}\,\ii)(2\ii-2\sqrt{2}\,\ii)=
        \ii^2\,(3+5\sqrt{2})(2-2\sqrt{2})= -8.25\) y su valor absoluto \(8.25\).
\end{ejemplos}

\begin{actividad}{3: Multiplicación}
  % Ejercicios previos
  \item \(\sqrt{-16}\cdot\sqrt{-25}\)
  \item \((2\sqrt{-2}+5\sqrt{-3})(\sqrt{-2}-4\sqrt{-3})\)
  \item \(\ii\cdot \ii \cdot \ii,\quad \ii^8,\quad \ii^{15}\)

  % ---- Nuevos ejercicios: Monomios ----
  \item \((7\ii)(-3\ii)\)
  \item \((12\ii)(5\ii)(-2\ii)\)
  \item \((-4\ii)(-6\ii)\)
  \item \((9\ii)(-9\ii)(\ii)\)
  \item \((2\ii)(3\ii)(4\ii)(5\ii)\)

  % ---- Nuevos ejercicios: Polinomios ----
  \item \((2\ii+3\ii)(4\ii+5\ii)\)
  \item \((\ii+2)(\ii-2)\)
  \item \((3\ii-1)(3\ii+1)\)
  \item \((\ii+1)(\ii+2)(\ii+3)\)
  \item \((2\ii-5)(2\ii+5)(\ii)\)
\end{actividad}

\begin{retos}
  \item Simplifica: \((3\ii)(4\ii)\).
  \item Evalúa: \((2\ii)(-7\ii)(5\ii)\).
  \item Determina: \(\ii^{2025}\).
\end{retos}
%! Author = alejandrogonzalezturrubiates
%! Date = 18/08/25

\begin{ejemplos}
  \item \(\dfrac{\sqrt{-84}}{\sqrt{-7}} = \dfrac{84\ii}{7\ii} = 12\).
  \item \(\dfrac{10\sqrt{-36}}{5\sqrt{-4}} = \dfrac{10\cdot 6\ii}{5\cdot 2\ii} = \dfrac{60\ii}{10\ii} = 6\).
\end{ejemplos}

\begin{actividad}{4: División}
  \item \(\dfrac{\sqrt{-27}}{\sqrt{-3}}\)
  \item \(\dfrac{\sqrt{-150}}{\sqrt{-3}}\)
  \item \(\dfrac{10\sqrt{-36}}{5\sqrt{-4}}\)
\end{actividad}
%! Author = alejandrogonzalezturrubiates
%! Date = 18/08/25

\section*{Ejercicios de aplicación}


% ==== Conversión a imaginarios (20) ====

\begin{ejercicio}
Convierte a forma con $i$: $\displaystyle \sqrt{-2}$.
\end{ejercicio}
\muestraSolucion{
$\displaystyle \sqrt{-2}=\sqrt{2}\,i.$
}

\begin{ejercicio}
Convierte a forma con $i$: $\displaystyle \sqrt{-18}$.
\end{ejercicio}
\muestraSolucion{
$\displaystyle \sqrt{-18}=3\sqrt{2}\,i.$
}

\begin{ejercicio}
Convierte a forma con $i$: $\displaystyle \sqrt{-75}$.
\end{ejercicio}
\muestraSolucion{
$\displaystyle \sqrt{-75}=5\sqrt{3}\,i.$
}

\begin{ejercicio}
Convierte a forma con $i$: $\displaystyle \sqrt{-0.81}$.
\end{ejercicio}
\muestraSolucion{
$\displaystyle \sqrt{-0.81}=0.9\,i.$
}

\begin{ejercicio}
Convierte a forma con $i$: $\displaystyle \sqrt{-\tfrac{1}{9}}$.
\end{ejercicio}
\muestraSolucion{
$\displaystyle \sqrt{-\tfrac{1}{9}}=\tfrac{1}{3}\,i.$
}

\begin{ejercicio}
Convierte a forma con $i$: $\displaystyle \sqrt{-48x^{2}}$.
\end{ejercicio}
\muestraSolucion{
$\displaystyle \sqrt{-48x^{2}}=4\sqrt{3}\,|x|\,i.$
}

\begin{ejercicio}
Convierte a forma con $i$: $\displaystyle \sqrt{-9a^{2}}$.
\end{ejercicio}
\muestraSolucion{
$\displaystyle \sqrt{-9a^{2}}=3\,|a|\,i.$
}

\begin{ejercicio}
Convierte a forma con $i$: $\displaystyle \sqrt{-4b^{4}}$.
\end{ejercicio}
\muestraSolucion{
$\displaystyle \sqrt{-4b^{4}}=2\,b^{2}\,i.$
}

\begin{ejercicio}
Convierte a forma con $i$: $\displaystyle \sqrt{-(x^{2}+y^{2})}$.
\end{ejercicio}
\muestraSolucion{
$\displaystyle \sqrt{-(x^{2}+y^{2})}=\sqrt{x^{2}+y^{2}}\,i.$
}

\begin{ejercicio}
Convierte a forma con $i$: $\displaystyle \sqrt{-98}$.
\end{ejercicio}
\muestraSolucion{
$\displaystyle \sqrt{-98}=7\sqrt{2}\,i.$
}

\begin{ejercicio}
Convierte a forma con $i$: $\displaystyle \sqrt{-32}$.
\end{ejercicio}
\muestraSolucion{
$\displaystyle \sqrt{-32}=4\sqrt{2}\,i.$
}

\begin{ejercicio}
Convierte a forma con $i$: $\displaystyle \sqrt{-72}$.
\end{ejercicio}
\muestraSolucion{
$\displaystyle \sqrt{-72}=6\sqrt{2}\,i.$
}

\begin{ejercicio}
Convierte a forma con $i$: $\displaystyle \sqrt{-\tfrac{16}{25}}$.
\end{ejercicio}
\muestraSolucion{
$\displaystyle \sqrt{-\tfrac{16}{25}}=\tfrac{4}{5}\,i.$
}

\begin{ejercicio}
Convierte a forma con $i$: $\displaystyle \sqrt{-200}$.
\end{ejercicio}
\muestraSolucion{
$\displaystyle \sqrt{-200}=10\sqrt{2}\,i.$
}

\begin{ejercicio}
Convierte a forma con $i$: $\displaystyle \sqrt{-12\,a^{2}}$.
\end{ejercicio}
\muestraSolucion{
$\displaystyle \sqrt{-12\,a^{2}}=2\sqrt{3}\,|a|\,i.$
}

\begin{ejercicio}
Convierte a forma con $i$: $\displaystyle \sqrt{-36\,m^{4}}$.
\end{ejercicio}
\muestraSolucion{
$\displaystyle \sqrt{-36\,m^{4}}=6\,m^{2}\,i.$
}

\begin{ejercicio}
Convierte a forma con $i$: $\displaystyle \sqrt{-18\,x^{4}}$.
\end{ejercicio}
\muestraSolucion{
$\displaystyle \sqrt{-18\,x^{4}}=3\sqrt{2}\,x^{2}\,i.$
}

\begin{ejercicio}
Convierte a forma con $i$: $\displaystyle \sqrt{-49\,y^{2}}$.
\end{ejercicio}
\muestraSolucion{
$\displaystyle \sqrt{-49\,y^{2}}=7\,|y|\,i.$
}

\begin{ejercicio}
Convierte a forma con $i$: $\displaystyle \sqrt{-3}$.
\end{ejercicio}
\muestraSolucion{
$\displaystyle \sqrt{-3}=\sqrt{3}\,i.$
}

\begin{ejercicio}
Convierte a forma con $i$: $\displaystyle \sqrt{-a^{2}b^{2}}$.
\end{ejercicio}
\muestraSolucion{
$\displaystyle \sqrt{-a^{2}b^{2}}=|ab|\,i.$
}


% =========================================================
\subsection{2) Suma y resta de imaginarios puros}

\begin{ejercicio}
Calcule $\displaystyle \sqrt{-9}+\sqrt{-16}$.
\end{ejercicio}
\muestraSolucion{
$\displaystyle \sqrt{-9}+\sqrt{-16}=3i+4i=7i.$
}

\begin{ejercicio}
Calcule $\displaystyle 2\sqrt{-25}-\sqrt{-49}$.
\end{ejercicio}
\muestraSolucion{
$\displaystyle 2\sqrt{-25}-\sqrt{-49}=2(5i)-7i=3i.$
}

\begin{ejercicio}
Calcule $\displaystyle \sqrt{-4}+\sqrt{-36}-\sqrt{-64}$.
\end{ejercicio}
\muestraSolucion{
$\displaystyle \sqrt{-4}+\sqrt{-36}-\sqrt{-64}=2i+6i-8i=0.$
}

\begin{ejercicio}
Calcule $\displaystyle 5\sqrt{-81}+3\sqrt{-16}$.
\end{ejercicio}
\muestraSolucion{
$\displaystyle 5\sqrt{-81}+3\sqrt{-16}=5(9i)+3(4i)=45i+12i=57i.$
}

\begin{ejercicio}
Calcule $\displaystyle \sqrt{-121}-2\sqrt{-49}$.
\end{ejercicio}
\muestraSolucion{
$\displaystyle \sqrt{-121}-2\sqrt{-49}=11i-14i=-3i.$
}

\begin{ejercicio}
Calcule $\displaystyle \sqrt{-1}+\sqrt{-4}+\sqrt{-9}+\sqrt{-16}$.
\end{ejercicio}
\muestraSolucion{
$\displaystyle i+2i+3i+4i=10i.$
}

\begin{ejercicio}
Calcule $\displaystyle \sqrt{-100}-\sqrt{-25}+\sqrt{-9}$.
\end{ejercicio}
\muestraSolucion{
$\displaystyle 10i-5i+3i=8i.$
}

\begin{ejercicio}
Calcule $\displaystyle 3\sqrt{-49}-\sqrt{-121}$.
\end{ejercicio}
\muestraSolucion{
$\displaystyle 3(7i)-11i=21i-11i=10i.$
}

\begin{ejercicio}
Calcule $\displaystyle \sqrt{-64}+\sqrt{-36}-\sqrt{-4}$.
\end{ejercicio}
\muestraSolucion{
$\displaystyle 8i+6i-2i=12i.$
}

\begin{ejercicio}
Calcule $\displaystyle -2\sqrt{-9}+\sqrt{-81}$.
\end{ejercicio}
\muestraSolucion{
$\displaystyle -2(3i)+9i=-6i+9i=3i.$
}

\begin{ejercicio}
Calcule $\displaystyle \sqrt{-m^2}+\sqrt{-4m^2}$.
\end{ejercicio}
\muestraSolucion{
$\displaystyle \sqrt{-m^2}+\sqrt{-4m^2}=mi+2mi=3mi.$
}

\begin{ejercicio}
Calcule $\displaystyle 7\sqrt{-x^2}-3\sqrt{-9x^2}$.
\end{ejercicio}
\muestraSolucion{
$\displaystyle 7xi-3(3xi)=7xi-9xi=-2xi.$
}

\begin{ejercicio}
Calcule $\displaystyle \sqrt{-49a^2}+\sqrt{-25a^2}$.
\end{ejercicio}
\muestraSolucion{
$\displaystyle \sqrt{-49a^2}+\sqrt{-25a^2}=7ai+5ai=12ai.$
}

\begin{ejercicio}
Calcule $\displaystyle \sqrt{-9b^2}-4\sqrt{-b^2}$.
\end{ejercicio}
\muestraSolucion{
$\displaystyle 3bi-4bi=-bi.$
}

\begin{ejercicio}
Calcule $\displaystyle (2\sqrt{-4}+3\sqrt{-9})-(\sqrt{-16})$.
\end{ejercicio}
\muestraSolucion{
$\displaystyle (2\cdot 2i+3\cdot 3i)-(4i)=(4i+9i)-4i=9i.$
}

\begin{ejercicio}
Calcule $\displaystyle 6\sqrt{-25}-2\sqrt{-100}$.
\end{ejercicio}
\muestraSolucion{
$\displaystyle 6(5i)-2(10i)=30i-20i=10i.$
}

\begin{ejercicio}
Calcule $\displaystyle \sqrt{-144}+\sqrt{-81}+\sqrt{-36}$.
\end{ejercicio}
\muestraSolucion{
$\displaystyle 12i+9i+6i=27i.$
}

\begin{ejercicio}
Calcule $\displaystyle \sqrt{-x^2}-\sqrt{-16x^2}+\sqrt{-25x^2}$.
\end{ejercicio}
\muestraSolucion{
$\displaystyle xi-4xi+5xi=2xi.$
}

\begin{ejercicio}
Calcule $\displaystyle 10\sqrt{-49}+2\sqrt{-121}-\sqrt{-4}$.
\end{ejercicio}
\muestraSolucion{
$\displaystyle 10(7i)+2(11i)-2i=70i+22i-2i=90i.$
}

\begin{ejercicio}
Calcule $\displaystyle -\sqrt{-64}+3\sqrt{-16}+5\sqrt{-4}$.
\end{ejercicio}
\muestraSolucion{
$\displaystyle -8i+12i+10i=14i.$
}

\subsection*{Ejercicios de Multiplicación con Imaginarios Puros}

\begin{ejercicio}
Calcule $\displaystyle (i)(i)$.
\end{ejercicio}
\muestraSolucion{
$\displaystyle (i)(i) = i^2 = -1.$
}

\begin{ejercicio}
Calcule $\displaystyle (3i)(4i)$.
\end{ejercicio}
\muestraSolucion{
$\displaystyle (3i)(4i) = 12i^2 = -12.$
}

\begin{ejercicio}
Calcule $\displaystyle (-5i)(2i)$.
\end{ejercicio}
\muestraSolucion{
$\displaystyle (-5i)(2i) = -10i^2 = 10.$
}

\begin{ejercicio}
Calcule $\displaystyle (7i)(-3i)$.
\end{ejercicio}
\muestraSolucion{
$\displaystyle (7i)(-3i) = -21i^2 = 21.$
}

\begin{ejercicio}
Calcule $\displaystyle (12i)(5i)(-2i)$.
\end{ejercicio}
\muestraSolucion{
$\displaystyle (12i)(5i)(-2i) = -120i^3 = 120i.$
}

\begin{ejercicio}
Calcule $\displaystyle (9i)(-9i)(i)$.
\end{ejercicio}
\muestraSolucion{
$\displaystyle (9i)(-9i)(i) = -81i^3 = -81(-i)=81i.$
}

\begin{ejercicio}
Calcule $\displaystyle (2i)(3i)(4i)(5i)$.
\end{ejercicio}
\muestraSolucion{
$\displaystyle (2i)(3i)(4i)(5i)=120i^4=120.$
}

\begin{ejercicio}
Calcule $\displaystyle (i)(i)(i)$.
\end{ejercicio}
\muestraSolucion{
$\displaystyle i^3=-i.$
}

\begin{ejercicio}
Calcule $\displaystyle i^8$.
\end{ejercicio}
\muestraSolucion{
$\displaystyle i^8=(i^4)^2=1.$
}

\begin{ejercicio}
Calcule $\displaystyle i^{15}$.
\end{ejercicio}
\muestraSolucion{
$\displaystyle i^{15}=i^{(15 \bmod 4)}=i^3=-i.$
}

\begin{ejercicio}
Calcule $\displaystyle (2i+3i)(4i+5i)$.
\end{ejercicio}
\muestraSolucion{
$\displaystyle (5i)(9i)=45i^2=-45.$
}

\begin{ejercicio}
Calcule $\displaystyle (i+2)(i-2)$.
\end{ejercicio}
\muestraSolucion{
$\displaystyle (i+2)(i-2)=i^2-4=-1-4=-5.$
}

\begin{ejercicio}
Calcule $\displaystyle (3i-1)(3i+1)$.
\end{ejercicio}
\muestraSolucion{
$\displaystyle (3i)^2-1^2=9i^2-1=-9-1=-10.$
}

\begin{ejercicio}
Calcule $\displaystyle (i+1)(i+2)(i+3)$.
\end{ejercicio}
\muestraSolucion{
$(i+1)(i+2)=i^2+3i+2=1+3i$, luego $(1+3i)(i+3)=1i+3+3i^2+9i=1i+3-3+9i=10i.$
}

\begin{ejercicio}
Calcule $\displaystyle (2i-5)(2i+5)(i)$.
\end{ejercicio}
\muestraSolucion{
$(2i-5)(2i+5)=4i^2-25=-4-25=-29$, luego $(-29)(i)=-29i.$
}

\begin{ejercicio}
Calcule $\displaystyle (4i)(-6i)(i)$.
\end{ejercicio}
\muestraSolucion{
$\displaystyle (4i)(-6i)(i)=-24i^3=24i.$
}

\begin{ejercicio}
Calcule $\displaystyle (5i)(5i)(-5i)$.
\end{ejercicio}
\muestraSolucion{
$\displaystyle 25i^2(-5i)=-25(5i)=-125i.$
}

\begin{ejercicio}
Calcule $\displaystyle (2+3i)(2-3i)$.
\end{ejercicio}
\muestraSolucion{
$\displaystyle (2+3i)(2-3i)=4-9i^2=4+9=13.$
}

\begin{ejercicio}
Calcule $\displaystyle (7i)(7i)(7i)(7i)$.
\end{ejercicio}
\muestraSolucion{
$\displaystyle (7^4)(i^4)=2401(1)=2401.$
}

\begin{ejercicio}
Calcule $\displaystyle (1+i)^2$.
\end{ejercicio}
\muestraSolucion{
$\displaystyle (1+i)^2=1+2i+i^2=2i.$
}




\section*{Ejercicios de División con Imaginarios Puros}

\begin{ejercicio}
Calcule $\displaystyle \frac{i}{i}$.
\end{ejercicio}
\muestraSolucion{
$\displaystyle \frac{i}{i} = 1.$
}

\begin{ejercicio}
Calcule $\displaystyle \frac{6i}{3i}$.
\end{ejercicio}
\muestraSolucion{
$\displaystyle \frac{6i}{3i} = 2.$
}

\begin{ejercicio}
Calcule $\displaystyle \frac{-8i}{2i}$.
\end{ejercicio}
\muestraSolucion{
$\displaystyle \frac{-8i}{2i} = -4.$
}

\begin{ejercicio}
Calcule $\displaystyle \frac{15i}{-5i}$.
\end{ejercicio}
\muestraSolucion{
$\displaystyle \frac{15i}{-5i} = -3.$
}

\begin{ejercicio}
Calcule $\displaystyle \frac{10i}{2i\cdoti}$.
\end{ejercicio}
\muestraSolucion{
$\displaystyle \frac{10i}{2i^2} = \frac{10i}{-2} = -5i.$
}

\begin{ejercicio}
Calcule $\displaystyle \frac{9i}{3}$.
\end{ejercicio}
\muestraSolucion{
$\displaystyle \frac{9i}{3} = 3i.$
}

\begin{ejercicio}
Calcule $\displaystyle \frac{12}{4i}$.
\end{ejercicio}
\muestraSolucion{
$\displaystyle \frac{12}{4i} = \frac{3}{i} = -3i.$
}

\begin{ejercicio}
Calcule $\displaystyle \frac{20}{5i}$.
\end{ejercicio}
\muestraSolucion{
$\displaystyle \frac{20}{5i} = \frac{4}{i} = -4i.$
}

\begin{ejercicio}
Calcule $\displaystyle \frac{7i}{14}$.
\end{ejercicio}
\muestraSolucion{
$\displaystyle \frac{7i}{14} = \tfrac{1}{2}i.$
}

\begin{ejercicio}
Calcule $\displaystyle \frac{i^5}{i^2}$.
\end{ejercicio}
\muestraSolucion{
$\displaystyle \frac{i^5}{i^2} = i^3 = -i.$
}

\begin{ejercicio}
Calcule $\displaystyle \frac{i^{10}}{i^6}$.
\end{ejercicio}
\muestraSolucion{
$\displaystyle \frac{i^{10}}{i^6} = i^4 = 1.$
}

\begin{ejercicio}
Calcule $\displaystyle \frac{i^{25}}{i^2}$.
\end{ejercicio}
\muestraSolucion{
$\displaystyle \frac{i^{25}}{i^2} = i^{23} = i^{3} = -i.$
}

\begin{ejercicio}
Calcule $\displaystyle \frac{(4i)(3i)}{6i}$.
\end{ejercicio}
\muestraSolucion{
$\displaystyle \frac{12i^2}{6i} = \frac{-12}{6i} = -2\cdot\frac{1}{i} = 2i.$
}

\begin{ejercicio}
Calcule $\displaystyle \frac{(2i)(5i)}{(10i)}$.
\end{ejercicio}
\muestraSolucion{
$\displaystyle \frac{10i^2}{10i} = \frac{-10}{10i} = -\frac{1}{i} = i.$
}

\begin{ejercicio}
Calcule $\displaystyle \frac{(-9i)(i)}{3i}$.
\end{ejercicio}
\muestraSolucion{
$\displaystyle \frac{-9i^2}{3i} = \frac{9}{3i} = \frac{3}{i} = -3i.$
}

\begin{ejercicio}
Calcule $\displaystyle \frac{50i}{-10}$.
\end{ejercicio}
\muestraSolucion{
$\displaystyle \frac{50i}{-10} = -5i.$
}

\begin{ejercicio}
Calcule $\displaystyle \frac{(6i)(2)}{3i}$.
\end{ejercicio}
\muestraSolucion{
$\displaystyle \frac{12i}{3i} = 4.$
}

\begin{ejercicio}
Calcule $\displaystyle \frac{(4+2i)}{2i}$.
\end{ejercicio}
\muestraSolucion{
$\displaystyle \frac{4}{2i} + \frac{2i}{2i} = \frac{2}{i}+1 = -2i+1.$
}

\begin{ejercicio}
Calcule $\displaystyle \frac{(3i)(-4i)}{12}$.
\end{ejercicio}
\muestraSolucion{
$\displaystyle \frac{-12}{12} = -1.$
}

\begin{ejercicio}
Calcule $\displaystyle \frac{100}{25i}$.
\end{ejercicio}
\muestraSolucion{
$\displaystyle \frac{100}{25i} = \frac{4}{i} = -4i.$
}
%! Author = alejandrogonzalezturrubiates
%! Date = 18/08/25

% Preamble
\section*{Módulo 2: Números Complejos (placeholder)}
% Objetivos, teoría de forma a+bi, operaciones, forma polar, etc.
%! Author = alejandrogonzalezturrubiates
%! Date = 09/09/25

%! Author = alejandrogonzalezturrubiates
%! Date = 04/09/25

\paragraph{Suma y resta de números complejos (50 ejercicios)}

\begin{ejercicio}
Calcule $(3+2i) + (4+5i)$.
\end{ejercicio}
\muestraSolucion{$7 + 7i$}

\begin{ejercicio}
Calcule $(7-3i) + (2+9i)$.
\end{ejercicio}
\muestraSolucion{$9 + 6i$}

\begin{ejercicio}
Calcule $(10+4i) - (6+7i)$.
\end{ejercicio}
\muestraSolucion{$4 - 3i$}

\begin{ejercicio}
Calcule $(-5+8i) + (12-2i)$.
\end{ejercicio}
\muestraSolucion{$7 + 6i$}

\begin{ejercicio}
Calcule $(9-11i) - (3-5i)$.
\end{ejercicio}
\muestraSolucion{$6 - 6i$}

\begin{ejercicio}
Calcule $(14+6i) + (-9+2i)$.
\end{ejercicio}
\muestraSolucion{$5 + 8i$}

\begin{ejercicio}
Calcule $(8+3i) - (5+9i)$.
\end{ejercicio}
\muestraSolucion{$3 - 6i$}

\begin{ejercicio}
Calcule $(2-7i) + (6+4i)$.
\end{ejercicio}
\muestraSolucion{$8 - 3i$}

\begin{ejercicio}
Calcule $(-10+5i) - (-3+2i)$.
\end{ejercicio}
\muestraSolucion{$-7 + 3i$}

\begin{ejercicio}
Calcule $(20-12i) + (15+7i)$.
\end{ejercicio}
\muestraSolucion{$35 - 5i$}

% ===== Bloque 2 =====
\begin{ejercicio}
Calcule $(11+9i) - (5+3i)$.
\end{ejercicio}
\muestraSolucion{$6 + 6i$}

\begin{ejercicio}
Calcule $(4+8i) + (7-2i)$.
\end{ejercicio}
\muestraSolucion{$11 + 6i$}

\begin{ejercicio}
Calcule $(-6-3i) + (9+5i)$.
\end{ejercicio}
\muestraSolucion{$3 + 2i$}

\begin{ejercicio}
Calcule $(15-7i) - (8-4i)$.
\end{ejercicio}
\muestraSolucion{$7 - 3i$}

\begin{ejercicio}
Calcule $(2+3i) + (-5+6i)$.
\end{ejercicio}
\muestraSolucion{$-3 + 9i$}

\begin{ejercicio}
Calcule $(12-8i) + (4+11i)$.
\end{ejercicio}
\muestraSolucion{$16 + 3i$}

\begin{ejercicio}
Calcule $(0+9i) - (3-7i)$.
\end{ejercicio}
\muestraSolucion{$-3 + 16i$}

\begin{ejercicio}
Calcule $(25+6i) - (18+2i)$.
\end{ejercicio}
\muestraSolucion{$7 + 4i$}

\begin{ejercicio}
Calcule $(5-10i) + (9+12i)$.
\end{ejercicio}
\muestraSolucion{$14 + 2i$}

\begin{ejercicio}
Calcule $(-11+7i) + (13-5i)$.
\end{ejercicio}
\muestraSolucion{$2 + 2i$}

% ===== Bloque 3 =====
\begin{ejercicio}
Calcule $(30-15i) + (20+5i)$.
\end{ejercicio}
\muestraSolucion{$50 - 10i$}

\begin{ejercicio}
Calcule $(7+3i) - (14+9i)$.
\end{ejercicio}
\muestraSolucion{$-7 - 6i$}

\begin{ejercicio}
Calcule $(18-4i) + (12-6i)$.
\end{ejercicio}
\muestraSolucion{$30 - 10i$}

\begin{ejercicio}
Calcule $(22+8i) - (17+3i)$.
\end{ejercicio}
\muestraSolucion{$5 + 5i$}

\begin{ejercicio}
Calcule $(4-11i) + (15+7i)$.
\end{ejercicio}
\muestraSolucion{$19 - 4i$}

\begin{ejercicio}
Calcule $(-9+12i) - (6-4i)$.
\end{ejercicio}
\muestraSolucion{$-15 + 16i$}

\begin{ejercicio}
Calcule $(19+0i) + (0+14i)$.
\end{ejercicio}
\muestraSolucion{$19 + 14i$}

\begin{ejercicio}
Calcule $(2+5i) - (7+9i)$.
\end{ejercicio}
\muestraSolucion{$-5 - 4i$}

\begin{ejercicio}
Calcule $(16-13i) + (8+3i)$.
\end{ejercicio}
\muestraSolucion{$24 - 10i$}

\begin{ejercicio}
Calcule $(9+7i) - (12-11i)$.
\end{ejercicio}
\muestraSolucion{$-3 + 18i$}

% ===== Bloque 4 =====
\begin{ejercicio}
Calcule $(5-6i) + (13+9i)$.
\end{ejercicio}
\muestraSolucion{$18 + 3i$}

\begin{ejercicio}
Calcule $(11+11i) - (3+4i)$.
\end{ejercicio}
\muestraSolucion{$8 + 7i$}

\begin{ejercicio}
Calcule $(20-8i) + (15-12i)$.
\end{ejercicio}
\muestraSolucion{$35 - 20i$}

\begin{ejercicio}
Calcule $(-4+9i) - (-7+3i)$.
\end{ejercicio}
\muestraSolucion{$3 + 6i$}

\begin{ejercicio}
Calcule $(10-15i) + (12+8i)$.
\end{ejercicio}
\muestraSolucion{$22 - 7i$}

\begin{ejercicio}
Calcule $(9+0i) + (-6+14i)$.
\end{ejercicio}
\muestraSolucion{$3 + 14i$}

\begin{ejercicio}
Calcule $(18-3i) - (9+7i)$.
\end{ejercicio}
\muestraSolucion{$9 - 10i$}

\begin{ejercicio}
Calcule $(21+11i) + (5-6i)$.
\end{ejercicio}
\muestraSolucion{$26 + 5i$}

\begin{ejercicio}
Calcule $(-8+13i) - (2+9i)$.
\end{ejercicio}
\muestraSolucion{$-10 + 4i$}

\begin{ejercicio}
Calcule $(30+0i) + (-15-12i)$.
\end{ejercicio}
\muestraSolucion{$15 - 12i$}

% ===== Bloque 5 =====
\begin{ejercicio}
Calcule $(40+22i) - (25+12i)$.
\end{ejercicio}
\muestraSolucion{$15 + 10i$}

\begin{ejercicio}
Calcule $(17+19i) + (9-8i)$.
\end{ejercicio}
\muestraSolucion{$26 + 11i$}

\begin{ejercicio}
Calcule $(14-6i) + (7+5i)$.
\end{ejercicio}
\muestraSolucion{$21 - i$}

\begin{ejercicio}
Calcule $(6+9i) - (12+4i)$.
\end{ejercicio}
\muestraSolucion{$-6 + 5i$}

\begin{ejercicio}
Calcule $(20+7i) + (13-3i)$.
\end{ejercicio}
\muestraSolucion{$33 + 4i$}

\begin{ejercicio}
Calcule $(11-14i) - (15-6i)$.
\end{ejercicio}
\muestraSolucion{$-4 - 8i$}

\begin{ejercicio}
Calcule $(-7+18i) + (8-9i)$.
\end{ejercicio}
\muestraSolucion{$1 + 9i$}

\begin{ejercicio}
Calcule $(9+5i) - (4+13i)$.
\end{ejercicio}
\muestraSolucion{$5 - 8i$}

\begin{ejercicio}
Calcule $(22-11i) + (17+15i)$.
\end{ejercicio}
\muestraSolucion{$39 + 4i$}

\begin{ejercicio}
Calcule $(16+24i) - (12+8i)$.
\end{ejercicio}
\muestraSolucion{$4 + 16i$}
%! Author = alejandrogonzalezturrubiates
%! Date = 09/09/25

%! Author = alejandrogonzalezturrubiates
%! Date = 05/09/25

\paragraph{Multiplicación de números complejos}

\begin{ejercicio}
Calcule $(3+2i)(1+4i)$.
\end{ejercicio}
\muestraSolucion{$3+12i+2i+8i^2 = 3+14i-8 = -5+14i$}

\begin{ejercicio}
Calcule $(5-3i)(5+3i)$.
\end{ejercicio}
\muestraSolucion{$25 - 9i^2 = 25+9 = 34$}

\begin{ejercicio}
Calcule $(7+2i)(4-6i)$.
\end{ejercicio}
\muestraSolucion{$28 - 42i + 8i - 12i^2 = 28 - 34i + 12 = 40 - 34i$}

\begin{ejercicio}
Calcule $(2+5i)(-3+7i)$.
\end{ejercicio}
\muestraSolucion{$-6+14i -15i+35i^2 = -6 - i -35 = -41 - i$}

\begin{ejercicio}
Calcule $(1+i)(1+i)(1+i)$.
\end{ejercicio}
\muestraSolucion{$(1+i)^2(1+i) = (1+2i+i^2)(1+i) = (2i)(1+i) = 2i+2i^2 = -2+2i$}

\begin{ejercicio}
Calcule $(8-3i)(-2-5i)$.
\end{ejercicio}
\muestraSolucion{$-16 -40i +6i+15i^2 = -16 -34i -15 = -31 -34i$}

\begin{ejercicio}
Calcule $(4+7i)(4-7i)$.
\end{ejercicio}
\muestraSolucion{$16 -49i^2 = 16+49 = 65$}

\begin{ejercicio}
Calcule $(3+4i)(2+6i)$.
\end{ejercicio}
\muestraSolucion{$6+18i+8i+24i^2 = 6+26i-24 = -18+26i$}

\begin{ejercicio}
Calcule $(9-2i)(-1+5i)$.
\end{ejercicio}
\muestraSolucion{$-9+45i+2i-10i^2 = -9+47i+10 = 1+47i$}

\begin{ejercicio}
Calcule $(2+i)(2-i)(3+4i)$.
\end{ejercicio}
\muestraSolucion{$(4- i^2)(3+4i) = 5(3+4i) = 15+20i$}
%! Author = alejandrogonzalezturrubiates
%! Date = 17/09/25

\paragraph{División de números complejos}

\begin{ejercicio}
Calcule $\displaystyle \frac{1+i}{1-i}$.
\end{ejercicio}
\muestraSolucion{
$\displaystyle \frac{1+i}{1-i} = \frac{(1+i)(1+i)}{(1-i)(1+i)} = \frac{1+2i+i^2}{1-(-1)} = \frac{2i}{2} = i.$
}

\begin{ejercicio}
Calcule $\displaystyle \frac{2-3i}{2+3i}$.
\end{ejercicio}
\muestraSolucion{
$\displaystyle \frac{2-3i}{2+3i} = \frac{(2-3i)(2-3i)}{(2+3i)(2-3i)}
= \frac{4-12i+9i^2}{4+9} = \frac{-5-12i}{13} = -\tfrac{5}{13}-\tfrac{12}{13}i.$
}

\begin{ejercicio}
Calcule $\displaystyle \frac{7+5i}{3+4i}$.
\end{ejercicio}
\muestraSolucion{
$\displaystyle \frac{7+5i}{3+4i} = \frac{(7+5i)(3-4i)}{3^2+4^2}
= \frac{21-28i+15i-20i^2}{25} = \frac{41-13i}{25} = \tfrac{41}{25}-\tfrac{13}{25}i.$
}

\begin{ejercicio}
Calcule $\displaystyle \frac{-2+11i}{4-i}$.
\end{ejercicio}
\muestraSolucion{
$\displaystyle \frac{-2+11i}{4-i} = \frac{(-2+11i)(4+i)}{4^2+(-1)^2}
= \frac{-8-2i+44i+11i^2}{17} = \frac{-19+42i}{17} = -\tfrac{19}{17}+\tfrac{42}{17}i.$
}

\begin{ejercicio}
Calcule $\displaystyle \frac{10+3i}{-6+8i}$.
\end{ejercicio}
\muestraSolucion{
$\displaystyle \frac{10+3i}{-6+8i} = \frac{(10+3i)(-6-8i)}{(-6)^2+8^2}
= \frac{-60-80i-18i-24i^2}{100} = \frac{-36-98i}{100} = -\tfrac{9}{25}-\tfrac{49}{50}i.$
}

\begin{ejercicio}
Calcule $\displaystyle \frac{15-4i}{9-12i}$.
\end{ejercicio}
\muestraSolucion{
$\displaystyle \frac{15-4i}{9-12i} = \frac{(15-4i)(9+12i)}{9^2+(-12)^2}
= \frac{135+180i-36i-48i^2}{225} = \frac{183+144i}{225} = \tfrac{61}{75}+\tfrac{16}{25}i.$
}

\begin{ejercicio}
Calcule $\displaystyle \frac{-5-7i}{8+15i}$.
\end{ejercicio}
\muestraSolucion{
$\displaystyle \frac{-5-7i}{8+15i} = \frac{(-5-7i)(8-15i)}{8^2+15^2}
= \frac{-40+75i-56i+105i^2}{289} = \frac{-145+19i}{289} = -\tfrac{145}{289}+\tfrac{19}{289}i.$
}

\begin{ejercicio}
Calcule $\displaystyle \frac{20+21i}{-7+24i}$.
\end{ejercicio}
\muestraSolucion{
$\displaystyle \frac{20+21i}{-7+24i} = \frac{(20+21i)(-7-24i)}{(-7)^2+24^2}
= \frac{-140-480i-147i-504i^2}{625} = \frac{364-627i}{625} = \tfrac{364}{625}-\tfrac{627}{625}i.$
}

\begin{ejercicio}
Calcule $\displaystyle \frac{13-5i}{13+5i}$.
\end{ejercicio}
\muestraSolucion{
$\displaystyle \frac{13-5i}{13+5i} = \frac{(13-5i)(13-5i)}{13^2+5^2}
= \frac{169-130i+25i^2}{194} = \frac{144-130i}{194} = \tfrac{72}{97}-\tfrac{65}{97}i.$
}

\begin{ejercicio}
Calcule $\displaystyle \frac{-9+40i}{3+4i}$.
\end{ejercicio}
\muestraSolucion{
$\displaystyle \frac{-9+40i}{3+4i} = \frac{(-9+40i)(3-4i)}{3^2+4^2}
= \frac{-27+36i+120i-160i^2}{25} = \frac{133+156i}{25} = \tfrac{133}{25}+\tfrac{156}{25}i.$
}
%! Author = alejandrogonzalezturrubiates
%! Date = 18/08/25

\section*{Módulo 3: División de Polinomios (placeholder)}
% Objetivos, método de Ruffini, división larga, aplicaciones.

% Placeholders de módulos futuros
% %! Author = alejandrogonzalezturrubiates
%! Date = 18/08/25

% Preamble
\section*{Módulo 2: Números Complejos (placeholder)}
% Objetivos, teoría de forma a+bi, operaciones, forma polar, etc.
% %! Author = alejandrogonzalezturrubiates
%! Date = 18/08/25

\section*{Módulo 3: División de Polinomios (placeholder)}
% Objetivos, método de Ruffini, división larga, aplicaciones.
% %! Author = alejandrogonzalezturrubiates
%! Date = 18/08/25

\section*{Módulo 4: Matrices (placeholder)}
% Objetivos, operaciones básicas, propiedades.
% %! Author = alejandrogonzalezturrubiates
%! Date = 18/08/25

\section*{Módulo 5: Determinantes (placeholder)}
% Objetivos, cofactores, propiedades, Sarrus/Cofactores.

%! Author = alejandrogonzalezturrubiates
%! Date = 18/08/25

\section*{Rúbrica general de evaluación}
\begin{tabular}{p{0.35\linewidth} p{0.6\linewidth}}
\toprule
\textbf{Criterio} & \textbf{Descripción} \\
\midrule
Procedimiento & Presenta pasos claros, justifica conversiones a \(i\) y uso de potencias (mod 4). \\
Exactitud & Resultados correctos y simplificados. \\
Claridad & Notación adecuada, orden y limpieza. \\
Reflexión & Explica errores comunes y cómo evitarlos. \\
\bottomrule
\end{tabular}

\end{document}