\documentclass[12pt]{article}
\usepackage[spanish]{babel}
\usepackage{amsmath, amssymb}
\usepackage[margin=2cm]{geometry}

% ====== Importamos tus macros ======
% Cambia la opción a [solucion] o [nosolucion] según versión
\usepackage{macros_simple}  % para examen sin resultados
% \usepackage[solucion]{macros}   % para examen con resultados

\begin{document}

\begin{center}
    \large \textbf{Facultad de Ingeniería Tampico}\\[6pt]
    \normalsize \textbf{Examen Parcial de Álgebra Lineal: Números Complejos}\\%[4pt]
    \normalsize Profesor: Dr. Alejandro González Turrubiates \\%[6pt]
    \normalsize Fecha: 23 de septiembre de 2025
\end{center}

%\vspace{0.3cm}
\noindent
Nombre: \underline{\hspace{8cm}} Matrícula: \underline{\hspace{3cm}} Grupo: \underline{\hspace{1cm}}
\vspace{0.5cm}

\hrule
\vspace{0.5cm}

\noindent
\textbf{Instrucciones:} Responda todos los ejercicios de manera clara y ordenada. Justifique cada procedimiento.

\vspace{0.5cm}

% =========================
% Reactivos
% =========================

\begin{ejercicio}
Calcule $(-6-3i) + (9+5i)$.
\end{ejercicio}
\muestraSolucion{$3 + 2i$ \vspace{-1cm}}

\vspace{1cm}

\begin{ejercicio}
Calcule $(15-7i) - (8-4i)$.
\end{ejercicio}
\muestraSolucion{$7 - 3i$ \vspace{-1cm}}

\vspace{1cm}

\begin{ejercicio}
Calcule $(5-3i)(5+3i)$.
\end{ejercicio}
\muestraSolucion{$25 - 9i^2 = 25+9 = 34$ \vspace{-1cm}}

\vspace{1cm}

\begin{ejercicio}
Calcule $(7+2i)(4-6i)$.
\end{ejercicio}
\muestraSolucion{$28 - 42i + 8i - 12i^2 = 28 - 34i + 12 = 40 - 34i$ \vspace{-2cm}}

\vspace{2cm}

\begin{ejercicio}
Calcule $\displaystyle \frac{7+5i}{3+4i}$.
\end{ejercicio}
\muestraSolucion{
$\displaystyle \frac{7+5i}{3+4i} = \frac{(7+5i)(3-4i)}{3^2+4^2}
= \frac{21-28i+15i-20i^2}{25} = \frac{41-13i}{25} = \tfrac{41}{25}-\tfrac{13}{25}i.$
}


\end{document}